\documentclass{article}
\usepackage{graphicx}
\usepackage[utf8]{inputenc}
\usepackage[T1]{fontenc}
\usepackage{fouriernc}
\usepackage[margin=1in]{geometry}
\usepackage{amsmath}
\begin{document}

\begin{titlepage}
	\centering 
	\scshape
	\vspace*{\baselineskip}
	\rule{\textwidth}{1.6pt}\vspace*{-\baselineskip}\vspace*{2pt}
	\rule{\textwidth}{0.4pt} 
	\vspace{0.75\baselineskip}
	
	{\Large CS 374 : Computational and Numerical Methods \\\vspace{0.75\baselineskip} Set 9}
	\vspace{0.75\baselineskip}
	
	\rule{\textwidth}{0.4pt}\vspace*{-\baselineskip}\vspace{3.2pt} 
	\rule{\textwidth}{1.6pt}
	
	\vspace{2\baselineskip}  
	The Gaussian Elimination Method
	
	\vspace*{3\baselineskip}
	
	\vspace{0.5\baselineskip} %originally 0.5
	
	{\scshape\large Purvil Mehta (201701073) \\ Bhargey Mehta (201701074) \\} 
	
	\vspace{1\baselineskip} 
	
	\textit{Dhirubhai Ambani Institute of Information and Communication Technology \\ Gandhinagar\\} 
	\vspace*{2\baselineskip}
	\today


\end{titlepage}

\newpage

\section*{Question 1}

\begin{Large}
$$
\begin{bmatrix}
x_1 \\
x_2 \\
x_3
\end{bmatrix}
=
\begin{bmatrix}
3.0000 \\
-1.6667\\
0.1111
\end{bmatrix}
$$
\end{Large}

\section*{Question 2}

\begin{Large}
$$
\begin{bmatrix}
x_1 \\
x_2 \\
x_3 \\
x_4
\end{bmatrix}
=
\begin{bmatrix}
0 \\
1 \\
-1 \\
0
\end{bmatrix}
$$
\end{Large}

\section*{Question 3}

\begin{Large}
$$
A =
\begin{bmatrix}
1 & 1 & -1\\
1 & 2 & -2\\
-2 & 1 & 1
\end{bmatrix}
$$
\end{Large}


\begin{Large}
$$
A^{-1} =
\begin{bmatrix}
    2.0 & -1  & 0\\
    1.5  & -0.5   & 0.5\\
    2.5   &-1.5  &  0.5
\end{bmatrix}
$$
\end{Large}



\end{document}